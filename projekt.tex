%==============================================================================
% tento soubor pouzijte jako zaklad
% this file should be used as a base for the thesis
% Autoři / Authors: 2008 Michal Bidlo, 2016 Jaroslav Dytrych
% Kontakt pro dotazy a připomínky: dytrych@fit.vutbr.cz
% Contact for questions and comments: dytrych@fit.vutbr.cz
%==============================================================================
% kodovaní: UTF-8 (zmena prikazem iconv, recode nebo cstocs)
% encoding: UTF-8 (you can change it by command iconv, recode or cstocs)
%------------------------------------------------------------------------------
% zpracování / processing: make, make pdf, make clean
%==============================================================================
% Soubory, které je nutné upravit: / Files which have to be edited:
%   projekt-20-literatura-bibliography.bib - literatura / bibliography
%   projekt-01-kapitoly-chapters.tex - obsah práce / the thesis content
%   projekt-30-prilohy-appendices.tex - přílohy / appendices
%==============================================================================
\documentclass[english]{fitthesis} % bez zadání - pro začátek práce, aby nebyl problém s překladem
%\documentclass[english]{fitthesis} % without assignment - for the work start to avoid compilation problem
%\documentclass[zadani]{fitthesis} % odevzdani do wisu - odkazy jsou barevné
%\documentclass[english,zadani]{fitthesis} % for submission to the IS FIT - links are color
%\documentclass[zadani,print]{fitthesis} % pro tisk - odkazy jsou černé
%\documentclass[zadani,cprint]{fitthesis} % pro barevný tisk - odkazy jsou černé, znak VUT barevný
%\documentclass[english,zadani,print]{fitthesis} % for the color print - links are black
%\documentclass[english,zadani,cprint]{fitthesis} % for the print - links are black, logo is color
% * Je-li prace psana v anglickem jazyce, je zapotrebi u tridy pouzit 
%   parametr english nasledovne:
%   If thesis is written in english, it is necessary to use 
%   parameter english as follows:
%      \documentclass[english]{fitthesis}
% * Je-li prace psana ve slovenskem jazyce, je zapotrebi u tridy pouzit 
%   parametr slovak nasledovne:
%      \documentclass[slovak]{fitthesis}

% Základní balíčky jsou dole v souboru šablony fitthesis.cls
% Basic packages are at the bottom of template file fitthesis.cls
%zde muzeme vlozit vlastni balicky / you can place own packages here

%---rm---------------
\renewcommand{\rmdefault}{lmr}%zavede Latin Modern Roman jako rm / set Latin Modern Roman as rm
%---sf---------------
\renewcommand{\sfdefault}{qhv}%zavede TeX Gyre Heros jako sf
%---tt------------
\renewcommand{\ttdefault}{lmtt}% zavede Latin Modern tt jako tt

% vypne funkci šablony, která automaticky nahrazuje uvozovky,
% aby nebyly prováděny nevhodné náhrady v popisech API apod.
% disables function of the template which replaces quotation marks
% to avoid unnecessary replacements in the API descriptions etc.
\csdoublequotesoff

% =======================================================================
% balíček "hyperref" vytváří klikací odkazy v pdf, pokud tedy použijeme pdflatex
% problém je, že balíček hyperref musí být uveden jako poslední, takže nemůže
% být v šabloně
% "hyperref" package create clickable links in pdf if you are using pdflatex.
% Problem is that this package have to be introduced as the last one so it 
% can not be placed in the template file.
\ifWis
\ifx\pdfoutput\undefined % nejedeme pod pdflatexem / we are not using pdflatex
\else
  \usepackage{color}
  \usepackage[unicode,colorlinks,hyperindex,plainpages=false,pdftex]{hyperref}
  \definecolor{links}{rgb}{0.4,0.5,0}
  \definecolor{anchors}{rgb}{1,0,0}
  \def\AnchorColor{anchors}
  \def\LinkColor{links}
  \def\pdfBorderAttrs{/Border [0 0 0] }  % bez okrajů kolem odkazů / without margins around links
  \pdfcompresslevel=9
\fi
\else % pro tisk budou odkazy, na které se dá klikat, černé / for the print clickable links will be black
\ifx\pdfoutput\undefined % nejedeme pod pdflatexem / we are not using pdflatex
\else
  \usepackage{color}
  \usepackage[unicode,colorlinks,hyperindex,plainpages=false,pdftex,urlcolor=black,linkcolor=black,citecolor=black]{hyperref}
  \definecolor{links}{rgb}{0,0,0}
  \definecolor{anchors}{rgb}{0,0,0}
  \def\AnchorColor{anchors}
  \def\LinkColor{links}
  \def\pdfBorderAttrs{/Border [0 0 0] } % bez okrajů kolem odkazů / without margins around links
  \pdfcompresslevel=9
\fi
\fi
% Řešení problému, kdy klikací odkazy na obrázky vedou za obrázek
% This solves the problems with links which leads after the picture
\usepackage[all]{hypcap}

% Informace o práci/projektu / Information about the thesis
%---------------------------------------------------------------------------
\projectinfo{
  %Prace / Thesis
  project=BP,            %typ prace BP/SP/DP/DR  / thesis type (SP = term project)
  year=2017,             %rok odevzdání / year of submission
  date=\today,           %datum odevzdani / submission date
  %Nazev prace / thesis title
  title.cs={Název práce},  %nazev prace v cestine ci slovenstine (dle zadani) / thesis title in czech language (according to assignment)
  title.en={Thesis title}, %nazev prace v anglictine / thesis title in english
  %Autor / Author
  author={Jméno Příjmení},   %cele jmeno a prijmeni autora / full name and surname of the author
  author.name={Jméno},   %jmeno autora (pro citaci) / author name (for reference) 
  author.surname={Příjmení},   %prijmeni autora (pro citaci) / author surname (for reference) 
  %author.title.p=Bc., %titul pred jmenem (nepovinne) / title before the name (optional)
  %author.title.a=PhD, %titul za jmenem (nepovinne) / title after the name (optional)
  %Ustav / Department
  department=UPGM, % doplnte prislusnou zkratku dle ustavu na zadani: UPSY/UIFS/UITS/UPGM
  %                  fill in appropriate abbreviation of the department according to assignment: UPSY/UIFS/UITS/UPGM
  %Skolitel / supervisor
  supervisor=Jméno Příjmení, %cele jmeno a prijmeni skolitele / full name and surname of the supervisor
  supervisor.name={Jméno},   %jmeno skolitele (pro citaci) / supervisor name (for reference) 
  supervisor.surname={Příjmení},   %prijmeni skolitele (pro citaci) / supervisor surname (for reference) 
  supervisor.title.p=Doc. RNDr.,   %titul pred jmenem (nepovinne) / title before the name (optional)
  supervisor.title.a={Ph.D.},    %titul za jmenem (nepovinne) / title after the name (optional)
  %Klicova slova, abstrakty, prohlaseni a podekovani je mozne definovat 
  %bud pomoci nasledujicich parametru nebo pomoci vyhrazenych maker (viz dale)
  %Keywords, abstracts, declaration and acknowledgement can be defined by following 
  %parameters or using dedicated macros (see below)
  %===========================================================================
  %Klicova slova / keywords
  %keywords.cs={Klíčová slova v českém jazyce.}, %klicova slova v ceskem ci slovenskem jazyce
  %                                              keywords in czech or slovak language
  %keywords.en={Klíčová slova v anglickém jazyce.}, %klicova slova v anglickem jazyce / keywords in english
  %Abstract
  %abstract.cs={Výtah (abstrakt) práce v českém jazyce.}, % abstrakt v ceskem ci slovenskem jazyce
  %                                                         abstract in czech or slovak language
  %abstract.en={Výtah (abstrakt) práce v anglickém jazyce.}, % abstrakt v anglickem jazyce / abstract in english
  %Prohlaseni / Declaration
  %declaration={Prohlašuji, že jsem tuto bakalářskou práci vypracoval samostatně pod vedením pana ...},
  %Podekovani (nepovinne) / Acknowledgement (optional)
  %acknowledgment={Zde je možné uvést poděkování vedoucímu práce a těm, kteří poskytli odbornou pomoc.} % nepovinne
  %acknowledgment={Here it is possible to express thanks to the supervisor and to the people which provided professional help.} % optional
}

%Abstrakt (cesky, slovensky ci anglicky) / Abstract (in czech, slovak or english)
\abstract[cs]{Do tohoto odstavce bude zapsán výtah (abstrakt) práce v českém (slovenském) jazyce.}
\abstract[en]{Do tohoto odstavce bude zapsán výtah (abstrakt) práce v anglickém jazyce.}

%Klicova slova (cesky, slovensky ci anglicky) / Keywords (in czech, slovak or english)
\keywords[cs]{Sem budou zapsána jednotlivá klíčová slova v českém (slovenském) jazyce, oddělená čárkami.}
\keywords[en]{Sem budou zapsána jednotlivá klíčová slova v anglickém jazyce, oddělená čárkami.}

%Prohlaseni (u anglicky psane prace anglicky, u slovensky psane prace slovensky)
%Declaration (for thesis in english should be in english)
\declaration{Prohlašuji, že jsem tuto bakalářskou práci vypracoval samostatně pod vedením pana X...
Další informace mi poskytli...
Uvedl jsem všechny literární prameny a publikace, ze kterých jsem čerpal.}

% \declaration{Hereby I declare that this bachelor's thesis was prepared as an original author’s work under the supervision of Mr. X
% The supplementary information was provided by Mr. Y
% All the relevant information sources, which were used during preparation of this thesis, are properly cited and included in the list of references.}

%Podekovani (nepovinne, nejlepe v jazyce prace) / Acknowledgement (optional, ideally in the language of the thesis)
\acknowledgment{V této sekci je možno uvést poděkování vedoucímu práce a těm, kteří poskytli odbornou pomoc
(externí zadavatel, konzultant, apod.).}
%\acknowledgment{Here it is possible to express thanks to the supervisor and to the people which provided professional help
%(external submitter, consultant, etc.).}

% řeší první/poslední řádek odstavce na předchozí/následující stránce
% solves first/last row of the paragraph on the previous/next page
\clubpenalty=10000
\widowpenalty=10000

\begin{document}
  % Vysazeni titulnich stran / Typesetting of the title pages
  % ----------------------------------------------
  \maketitle
  % Obsah
  % ----------------------------------------------
  \setlength{\parskip}{0pt}

  {\hypersetup{hidelinks}\tableofcontents}
  
  % Seznam obrazku a tabulek (pokud prace obsahuje velke mnozstvi obrazku, tak se to hodi)
  % List of figures and list of tables (if the thesis contains a lot of pictures, it is good)
  \ifczech
    \renewcommand\listfigurename{Seznam obrázků}
  \fi
  \ifslovak
    \renewcommand\listfigurename{Zoznam obrázkov}
  \fi
  % \listoffigures
  
  \ifczech
    \renewcommand\listtablename{Seznam tabulek}
  \fi
  \ifslovak
    \renewcommand\listtablename{Zoznam tabuliek}
  \fi
  % \listoftables 

  \ifODSAZ
    \setlength{\parskip}{0.5\bigskipamount}
  \else
    \setlength{\parskip}{0pt}
  \fi

  % vynechani stranky v oboustrannem rezimu
  % Skip the page in the two-sided mode
  \iftwoside
    \cleardoublepage
  \fi

  % Text prace / Thesis text
  % ----------------------------------------------
  %=========================================================================
% (c) Michal Bidlo, Bohuslav Křena, 2008

\chapter{Úvod}
Abychom mohli napsat odborný text jasně a~srozumitelně, musíme splnit několik základních předpokladů:
\begin{itemize}
\item Musíme mít co říci,
\item musíme vědět, komu to chceme říci,
\item musíme si dokonale promyslet obsah,
\item musíme psát strukturovaně. 
\end{itemize}

Tyto a další pokyny jsou dostupné též na školních internetových stránkách \cite{fitWeb}.

Přehled základů typografie a tvorby dokumentů s využitím systému \LaTeX je 
uveden v~\cite{Rybicka}.

\section{Musíme mít co říci}
Dalším důležitým předpokladem dobrého psaní je {\it psát pro někoho}. Píšeme-li si poznámky sami pro sebe, píšeme je jinak než výzkumnou zprávu, článek, diplomovou práci, knihu nebo dopis. Podle předpokládaného čtenáře se rozhodneme pro způsob psaní, rozsah informace a~míru detailů.

\section{Musíme vědět, komu to chceme říci}
Dalším důležitým předpokladem dobrého psaní je psát pro někoho. Píšeme-li si poznámky sami pro sebe, píšeme je jinak než výzkumnou zprávu, článek, diplomovou práci, knihu nebo dopis. Podle předpokládaného čtenáře se rozhodneme pro způsob psaní, rozsah informace a~míru detailů.

\section{Musíme si dokonale promyslet obsah}
Musíme si dokonale promyslet a~sestavit obsah sdělení a~vytvořit pořadí, v~jakém chceme čtenáři své myšlenky prezentovat. 
Jakmile víme, co chceme říci a~komu, musíme si rozvrhnout látku. Ideální je takové rozvržení, které tvoří logicky přesný a~psychologicky stravitelný celek, ve kterém je pro všechno místo a~jehož jednotlivé části do sebe přesně zapadají. Jsou jasné všechny souvislosti a~je zřejmé, co kam patří.

Abychom tohoto cíle dosáhli, musíme pečlivě organizovat látku. Rozhodneme, co budou hlavní kapitoly, co podkapitoly a~jaké jsou mezi nimi vztahy. Diagramem takové organizace je graf, který je velmi podobný stromu, ale ne řetězci. Při organizaci látky je stejně důležitá otázka, co do osnovy zahrnout, jako otázka, co z~ní vypustit. Příliš mnoho podrobností může čtenáře právě tak odradit jako žádné detaily.

Výsledkem této etapy je osnova textu, kterou tvoří sled hlavních myšlenek a~mezi ně zařazené detaily.

\section{Musíme psát strukturovaně} 
Musíme začít psát strukturovaně a~současně pracujeme na co nejsrozumitelnější formě, včetně dobrého slohu a~dokonalého značení. 
Máme-li tedy myšlenku, představu o~budoucím čtenáři, cíl a~osnovu textu, můžeme začít psát. Při psaní prvního konceptu se snažíme zaznamenat všechny své myšlenky a~názory vztahující se k~jednotlivým kapitolám a~podkapitolám. Každou myšlenku musíme vysvětlit, popsat a~prokázat. Hlavní myšlenku má vždy vyjadřovat hlavní věta a~nikoliv věta vedlejší.

I k~procesu psaní textu přistupujeme strukturovaně. Současně s~tím, jak si ujasňujeme strukturu písemné práce, vytváříme kostru textu, kterou postupně doplňujeme. Využíváme ty prostředky DTP programu, které podporují strukturovanou stavbu textu (předdefinované typy pro nadpisy a~bloky textu). 


\chapter{Několik formálních pravidel}
Naším cílem je vytvořit jasný a~srozumitelný text. Vyjadřujeme se proto přesně, píšeme dobrou češtinou (nebo zpravidla angličtinou) a~dobrým slohem podle obecně přijatých zvyklostí. Text má upravit čtenáři cestu k~rychlému pochopení problému, předvídat jeho obtíže a~předcházet jim. Dobrý sloh předpokládá bezvadnou gramatiku, správnou interpunkci a~vhodnou volbu slov. Snažíme se, aby náš text nepůsobil příliš jednotvárně používáním malého výběru slov a~tím, že některá zvlášť oblíbená slova používáme příliš často. Pokud používáme cizích slov, je samozřejmým předpokladem, že známe jejich přesný význam. Ale i~českých slov musíme používat ve správném smyslu. Např. platí jistá pravidla při používání slova {\it zřejmě}. Je {\it zřejmé} opravdu zřejmé? A~přesvědčili jsme se, zda to, co je {\it zřejmé} opravdu platí? Pozor bychom si měli dát i~na příliš časté používání zvratného se. Například obratu {\it dokázalo se}, že\ldots{} zásadně nepoužíváme. Není špatné používat autorského {\it my}, tím předpokládáme, že něco řešíme, nebo například zobecňujeme spolu se čtenářem. V~kvalifikačních pracích použijeme autorského {\it já} (například když vymezujeme podíl vlastní práce vůči převzatému textu), ale v~běžném textu se nadměrné používání první osoby jednotného čísla nedoporučuje.

Za pečlivý výběr stojí i~symbolika, kterou používáme ke {\it značení}. Máme tím na mysli volbu zkratek a~symbolů používaných například pro vyjádření typů součástek, pro označení hlavních činností programu, pro pojmenování ovládacích kláves na klávesnici, pro pojmenování proměnných v~matematických formulích a~podobně. Výstižné a~důsledné značení může čtenáři při četbě textu velmi pomoci. Je vhodné uvést seznam značení na začátku textu. Nejen ve značení, ale i~v~odkazech a~v~celkové tiskové úpravě je důležitá důslednost.

S tím souvisí i~pojem z~typografie nazývaný {\it vyznačování}. Zde máme na mysli způsob sazby textu pro jeho zvýraznění. Pro zvolené značení by měl být zvolen i~způsob vyznačování v~textu. Tak například klávesy mohou být umístěny do obdélníčku, identifikátory ze~zdrojového textu mohou být vypisovány {\tt písmem typu psací stroj} a~podobně.

Uvádíme-li některá fakta, neskrýváme jejich původ a~náš vztah k~nim. Když něco tvrdíme, vždycky výslovně uvedeme, co z~toho bylo dokázáno, co teprve bude dokázáno v~našem textu a~co přebíráme z~literatury s~uvedením odkazu na příslušný zdroj. V~tomto směru nenecháváme čtenáře nikdy na pochybách, zda jde o~myšlenku naši nebo převzatou z~literatury.

Nikdy neplýtváme čtenářovým časem výkladem triviálních a~nepodstatných informací. Neuvádíme rovněž několikrát totéž jen jinými slovy. Při pozdějších úpravách textu se nám může některá dříve napsaná pasáž jevit jako zbytečně podrobná nebo dokonce zcela zbytečná. Vypuštění takové pasáže nebo alespoň její zestručnění přispěje k~lepší čitelnosti práce! Tento krok ale vyžaduje odvahu zahodit čas, který jsme jejímu vytvoření věnovali. 


\chapter{Nikdy to nebude naprosto dokonalé}
Když jsme už napsali vše, o~čem jsme přemýšleli, uděláme si den nebo dva dny volna a~pak si přečteme sami rukopis znovu. Uděláme ještě poslední úpravy a~skončíme. Jsme si vědomi toho, že vždy zůstane něco nedokončeno, vždy existuje lepší způsob, jak něco vysvětlit, ale každá etapa úprav musí být konečná.


\chapter{Typografické a~jazykové zásady}
Při tisku odborného textu typu {\it technická zpráva} (anglicky {\it technical report}), ke kterému patří například i~text kvalifikačních prací, se často volí formát A4 a~často se tiskne pouze po~jedné straně papíru. V~takovém případě volte levý okraj všech stránek o~něco větší než pravý -- v~tomto místě budou papíry svázány a~technologie vazby si tento požadavek vynucuje. Při vazbě s~pevným hřbetem by se levý okraj měl dělat o~něco širší pro tlusté svazky, protože se stránky budou hůře rozevírat a~levý okraj se tak bude oku méně odhalovat.

Horní a~spodní okraj volte stejně veliký, případně potištěnou část posuňte mírně nahoru (horní okraj menší než dolní). Počítejte s~tím, že při vazbě budou okraje mírně oříznuty.

Pro sazbu na stránku formátu A4 je vhodné používat pro základní text písmo stupně (velikosti) 11 bodů. Volte šířku sazby 15 až 16 centimetrů a~výšku 22 až 23 centimetrů (včetně případných hlaviček a~patiček). Proklad mezi řádky se volí 120 procent stupně použitého základního písma, což je optimální hodnota pro rychlost čtení souvislého textu. V~případě použití systému LaTeX ponecháme implicitní nastavení. Při psaní kvalifikační práce se řiďte příslušnými závaznými požadavky.

Stupeň písma u~nadpisů různé úrovně volíme podle standardních typografických pravidel. 
Pro všechny uvedené druhy nadpisů se obvykle používá polotučné nebo tučné písmo (jednotně buď všude polotučné nebo všude tučné). Proklad se volí tak, aby se následující text běžných odstavců sázel pokud možno na {\it pevný rejstřík}, to znamená jakoby na linky s~předem definovanou a~pevnou roztečí.

Uspořádání jednotlivých částí textu musí být přehledné a~logické. Je třeba odlišit názvy kapitol a~podkapitol -- píšeme je malými písmeny kromě velkých začátečních písmen. U~jednotlivých odstavců textu odsazujeme první řádek odstavce asi o~jeden až dva čtverčíky (vždy o~stejnou, předem zvolenou hodnotu), tedy přibližně o~dvě šířky velkého písmene M základního textu. Poslední řádek předchozího odstavce a~první řádek následujícího odstavce se v~takovém případě neoddělují svislou mezerou. Proklad mezi těmito řádky je stejný jako proklad mezi řádky uvnitř odstavce.

Při vkládání obrázků volte jejich rozměry tak, aby nepřesáhly oblast, do které se tiskne text (tj. okraje textu ze všech stran). Pro velké obrázky vyčleňte samostatnou stránku. Obrázky nebo tabulky o~rozměrech větších než A4 umístěte do písemné zprávy formou skládanky všité do přílohy nebo vložené do záložek na zadní desce.

Obrázky i~tabulky musí být pořadově očíslovány. Číslování se volí buď průběžné v~rámci celého textu, nebo -- což bývá praktičtější -- průběžné v~rámci kapitoly. V~druhém případě se číslo tabulky nebo obrázku skládá z~čísla kapitoly a~čísla obrázku/tabulky v~rámci kapitoly -- čísla jsou oddělena tečkou. Čísla podkapitol nemají na číslování obrázků a~tabulek žádný vliv.

Tabulky a~obrázky používají své vlastní, nezávislé číselné řady. Z toho vyplývá, že v~odkazech uvnitř textu musíme kromě čísla udat i~informaci o~tom, zda se jedná o~obrázek či tabulku (například \uv{\ldots {\it viz tabulka 2.7} \ldots}). Dodržování této zásady je ostatně velmi přirozené.

Pro odkazy na stránky, na čísla kapitol a~podkapitol, na čísla obrázků a~tabulek a~v~dalších podobných příkladech využíváme speciálních prostředků DTP programu, které zajistí vygenerování správného čísla i~v~případě, že se text posune díky změnám samotného textu nebo díky úpravě parametrů sazby. Příkladem takového prostředku v~systému LaTeX je odkaz na číslo odpovídající umístění značky v~textu, například návěští ($\backslash${\tt ref\{navesti\}} -- podle umístění návěští se bude jednat o~číslo kapitoly, podkapitoly, obrázku, tabulky nebo podobného číslovaného prvku), na stránku, která obsahuje danou značku ($\backslash${\tt pageref\{navesti\}}), nebo na literární odkaz ($\backslash${\tt cite\{identifikator\}}).

Rovnice, na které se budeme v~textu odvolávat, opatříme pořadovými čísly při pravém okraji příslušného řádku. Tato pořadová čísla se píší v~kulatých závorkách. Číslování rovnic může být průběžné v~textu nebo v~jednotlivých kapitolách.

Jste-li na pochybách při sazbě matematického textu, snažte se dodržet způsob sazby definovaný systémem LaTeX. Obsahuje-li vaše práce velké množství matematických formulí, doporučujeme dát přednost použití systému LaTeX.

Mezeru neděláme tam, kde se spojují číslice s~písmeny v~jedno slovo nebo v~jeden znak -- například {\it 25krát}.

Členicí (interpunkční) znaménka tečka, čárka, středník, dvojtečka, otazník a~vykřičník, jakož i~uzavírací závorky a~uvozovky se přimykají k~předcházejícímu slovu bez mezery. Mezera se dělá až za nimi. To se ovšem netýká desetinné čárky (nebo desetinné tečky). Otevírací závorka a~přední uvozovky se přimykají k~následujícímu slovu a~mezera se vynechává před nimi -- (takto) a~\uv{takto}.

Pro spojovací a~rozdělovací čárku a~pomlčku nepoužíváme stejný znak. Pro pomlčku je vyhrazen jiný znak (delší). V~systému TeX (LaTeX) se spojovací čárka zapisuje jako jeden znak \uv{pomlčka} (například \uv{Brno-město}), pro sázení textu ve smyslu intervalu nebo dvojic, soupeřů a~podobně se ve zdrojovém textu používá dvojice znaků \uv{pomlčka} (například \uv{zápas Sparta -- Slavie}; \uv{cena 23--25 korun}), pro výrazné oddělení části věty, pro výrazné oddělení vložené věty, pro vyjádření nevyslovené myšlenky a~v~dalších situacích (viz Pravidla českého pravopisu) se používá nejdelší typ pomlčky, která se ve zdrojovém textu zapisuje jako trojice znaků \uv{pomlčka} (například \uv{Další pojem --- jakkoliv se může zdát nevýznamný --- bude neformálně definován v~následujícím odstavci.}). Při sazbě matematického mínus se při sazbě používá rovněž odlišný znak. V~systému TeX je ve zdrojovém textu zapsán jako normální mínus (tj. znak \uv{pomlčka}). Sazba v~matematickém prostředí, kdy se vzoreček uzavírá mezi dolary, zajistí vygenerování správného výstupu.

Lomítko se píše bez mezer. Například školní rok 2008/2009.

Pravidla pro psaní zkratek jsou uvedena v~Pravidlech českého pravopisu \cite{Pravidla}. I~z~jiných důvodů je vhodné, abyste tuto knihu měli po ruce. 


\section{Co to je normovaná stránka?}
Pojem {\it normovaná stránka} se vztahuje k~posuzování objemu práce, nikoliv k~počtu vytištěných listů. Z historického hlediska jde o~počet stránek rukopisu, který se psal psacím strojem na speciální předtištěné formuláře při dodržení průměrné délky řádku 60 znaků a~při 30 řádcích na stránku rukopisu. Vzhledem k~zápisu korekturních značek se používalo řádkování 2 (ob jeden řádek). Tyto údaje (počet znaků na řádek, počet řádků a~proklad mezi nimi) se nijak nevztahují ke konečnému vytištěnému výsledku. Používají se pouze pro posouzení rozsahu. Jednou normovanou stránkou se tedy rozumí 60*30 = 1800 znaků. Obrázky zařazené do textu se započítávají do rozsahu písemné práce odhadem jako množství textu, které by ve výsledném dokumentu potisklo stejně velkou plochu.

Orientační rozsah práce v~normostranách lze v~programu Microsoft Word zjistit pomocí funkce {\it Počet slov} v~menu {\it Nástroje}, když hodnotu {\it Znaky (včetně mezer)} vydělíte konstantou 1800. Do rozsahu práce se započítává pouze text uvedený v~jádru práce. Části jako abstrakt, klíčová slova, prohlášení, obsah, literatura nebo přílohy se do rozsahu práce nepočítají. Je proto nutné nejdříve označit jádro práce a~teprve pak si nechat spočítat počet znaků. Přibližný rozsah obrázků odhadnete ručně. Podobně lze postupovat i~při použití OpenOffice. Při použití systému LaTeX pro sazbu je situace trochu složitější. Pro hrubý odhad počtu normostran lze využít součet velikostí zdrojových souborů práce podělený konstantou cca 2000 (normálně bychom dělili konstantou 1800, jenže ve zdrojových souborech jsou i~vyznačovací příkazy, které se do rozsahu nepočítají). Pro přesnější odhad lze pak vyextrahovat holý text z~PDF (např. metodou cut-and-paste nebo {\it Save as Text\ldots}) a~jeho velikost podělit konstantou 1800. 


\chapter{Závěr}
Závěrečná kapitola obsahuje zhodnocení dosažených výsledků se zvlášť vyznačeným vlastním přínosem studenta. Povinně se zde objeví i zhodnocení z pohledu dalšího vývoje projektu, student uvede náměty vycházející ze zkušeností s řešeným projektem a uvede rovněž návaznosti na právě dokončené projekty.

%=========================================================================
 % viz. obsah.tex / see obsah.tex

  % Pouzita literatura / Bibliography
  % ----------------------------------------------
\ifslovak
  \makeatletter
  \def\@openbib@code{\addcontentsline{toc}{chapter}{Literatúra}}
  \makeatother
  \bibliographystyle{bib-styles/czechiso}
\else
  \ifczech
    \makeatletter
    \def\@openbib@code{\addcontentsline{toc}{chapter}{Literatura}}
    \makeatother
    \bibliographystyle{bib-styles/czechiso}
  \else 
    \makeatletter
    \def\@openbib@code{\addcontentsline{toc}{chapter}{Bibliography}}
    \makeatother
    \bibliographystyle{bib-styles/englishiso}
  %  \bibliographystyle{alpha}
  \fi
\fi
  \begin{flushleft}
  \bibliography{projekt-20-literatura-bibliography}
  \end{flushleft}

  % vynechani stranky v oboustrannem rezimu
  % Skip the page in the two-sided mode
  \iftwoside
    \cleardoublepage
  \fi

  % Prilohy / Appendices
  % ---------------------------------------------
  \appendix
\ifczech
  \renewcommand{\appendixpagename}{Přílohy}
  \renewcommand{\appendixtocname}{Přílohy}
  \renewcommand{\appendixname}{Příloha}
\fi
\ifslovak
  \renewcommand{\appendixpagename}{Prílohy}
  \renewcommand{\appendixtocname}{Prílohy}
  \renewcommand{\appendixname}{Príloha}
\fi
%  \appendixpage

% vynechani stranky v oboustrannem rezimu
% Skip the page in the two-sided mode
%\iftwoside
%  \cleardoublepage
%\fi
  
\ifslovak
%  \section*{Zoznam príloh}
%  \addcontentsline{toc}{section}{Zoznam príloh}
\else
  \ifczech
%    \section*{Seznam příloh}
%    \addcontentsline{toc}{section}{Seznam příloh}
  \else
%    \section*{List of Appendices}
%    \addcontentsline{toc}{section}{List of Appendices}
  \fi
\fi
  \startcontents[chapters]
  \setlength{\parskip}{0pt}
  % seznam příloh / list of appendices
  % \printcontents[chapters]{l}{0}{\setcounter{tocdepth}{2}}
  
  \ifODSAZ
    \setlength{\parskip}{0.5\bigskipamount}
  \else
    \setlength{\parskip}{0pt}
  \fi
  
  % vynechani stranky v oboustrannem rezimu
  \iftwoside
    \cleardoublepage
  \fi
  % Tento soubor nahraďte vlastním souborem s přílohami (nadpisy níže jsou pouze pro příklad)
% This file should be replaced with your file with an appendices (headings below are examples only)

% Umístění obsahu paměťového média do příloh je vhodné konzultovat s vedoucím
% Placing of table of contents of the memory media here should be consulted with a supervisor
%\chapter{Obsah přiloženého paměťového média}

%\chapter{Manuál}

%\chapter{Konfigurační soubor} % Configuration file

%\chapter{RelaxNG Schéma konfiguračního souboru} % Scheme of RelaxNG configuration file

%\chapter{Plakát} % poster

\chapter{Jak pracovat s touto šablonou}
\label{jak}

V této kapitole je uveden popis jednotlivých částí šablony, po kterém následuje stručný návod, jak s touto šablonou pracovat. 

Jedná se o přechodnou verzi šablony. Nová verze bude zveřejněna do~konce roku 2017 a~bude navíc obsahovat nové pokyny ke správnému využití šablony, závazné pokyny k~vypracování bakalářských a diplomových prací (rekapitulace pokynů, které jsou dostupné na~webu) a nezávazná doporučení od vybraných vedoucích, která již teď najdete na~webu (viz odkazy v souboru s literaturou). Jediné soubory, které se v nové verzi změní, budou \texttt{projekt-01-kapitoly-chapters.tex} a \texttt{projekt-30-prilohy-appendices.tex}, jejichž obsah každý student vymaže a nahradí vlastním. Šablonu lze tedy bez problémů využít i~v~současné verzi.

\section*{Popis částí šablony}

Po rozbalení šablony naleznete následující soubory a adresáře:
\begin{DESCRIPTION}
  \item [bib-styles] Styly literatury (viz níže). 
  \item [obrazky-figures] Adresář pro Vaše obrázky. Nyní obsahuje placeholder.pdf (tzv. TODO obrázek, který lze použít jako pomůcku při tvorbě technické zprávy), který se s prací neodevzdává. Název adresáře je vhodné zkrátit, aby byl jen ve zvoleném jazyce.
  \item [template-fig] Obrázky šablony (znak VUT).
  \item [fitthesis.cls] Šablona (definice vzhledu).
  \item [Makefile] Makefile pro překlad, počítání normostran, sbalení apod. (viz níže).
  \item [projekt-01-kapitoly-chapters.tex] Soubor pro Váš text (obsah nahraďte).
  \item [projekt-20-literatura-bibliography.bib] Seznam literatury (viz níže).
  \item [projekt-30-prilohy-appendices.tex] Soubor pro přílohy (obsah nahraďte).
  \item [projekt.tex] Hlavní soubor práce -- definice formálních částí.
\end{DESCRIPTION}

Výchozí styl literatury (czechiso) je od Ing. Martínka, přičemž anglická verze (englishiso) je jeho překladem s drobnými modifikacemi. Oproti normě jsou v něm určité odlišnosti, ale na FIT je dlouhodobě akceptován. Alternativně můžete využít styl od Ing. Radima Loskota nebo od Ing. Radka Pyšného\footnote{BP Ing. Radka Pyšného \url{http://www.fit.vutbr.cz/study/DP/BP.php?id=7848}}. Alternativní styly obsahují určitá vylepšení, ale zatím nebyly řádně otestovány větším množstvím uživatelů. Lze je považovat za beta verze pro zájemce, kteří svoji práci chtějí mít dokonalou do detailů a neváhají si nastudovat detaily správného formátování citací, aby si mohli ověřit, že je vysázený výsledek v pořádku.

Makefile kromě překladu do PDF nabízí i další funkce:
\begin{itemize}
  \item přejmenování souborů (viz níže),
  \item počítání normostran,
  \item spuštění vlny pro doplnění nezlomitelných mezer,
  \item sbalení výsledku pro odeslání vedoucímu ke kontrole (zkontrolujte, zda sbalí všechny Vámi přidané soubory, a případně doplňte).
\end{itemize}

Nezapomeňte, že vlna neřeší všechny nezlomitelné mezery. Vždy je třeba manuální kontrola, zda na konci řádku nezůstalo něco nevhodného -- viz Internetová jazyková příručka\footnote{Internetová jazyková příručka \url{http://prirucka.ujc.cas.cz/?id=880}}.

\paragraph {Pozor na číslování stránek!} Pokud má obsah 2 strany a na 2. jsou jen \uv{Přílohy} a~\uv{Seznam příloh} (ale žádná příloha tam není), z nějakého důvodu se posune číslování stránek o 1 (obsah \uv{nesedí}). Stejný efekt má, když je na 2. či 3. stránce obsahu jen \uv{Literatura} a~je možné, že tohoto problému lze dosáhnout i jinak. Řešení je několik (od~úpravy obsahu, přes nastavení počítadla až po sofistikovanější metody). \textbf{Před odevzdáním proto vždy překontrolujte číslování stran!}


\section*{Doporučený postup práce se šablonou}

\begin{enumerate}
  \item \textbf{Zkontrolujte, zda máte aktuální verzi šablony.} Máte-li šablonu z předchozího roku, na stránkách fakulty již může být novější verze šablony s~aktualizovanými informacemi, opravenými chybami apod.
  \item \textbf{Zvolte si jazyk}, ve kterém budete psát svoji technickou zprávu (česky, slovensky nebo anglicky) a svoji volbu konzultujte s vedoucím práce (nebyla-li dohodnuta předem). Pokud Vámi zvoleným jazykem technické zprávy není čeština, nastavte příslušný parametr šablony v souboru projekt.tex (např.: \verb|documentclass[english]{fitthesis}| a přeložte prohlášení a poděkování do~angličtiny či slovenštiny.
  \item \textbf{Přejmenujte soubory.} Po rozbalení je v šabloně soubor \texttt{projekt.tex}. Pokud jej přeložíte, vznikne PDF s technickou zprávou pojmenované \texttt{projekt.pdf}. Když vedoucímu více studentů pošle \texttt{projekt.pdf} ke kontrole, musí je pracně přejmenovávat. Proto je vždy vhodné tento soubor přejmenovat tak, aby obsahoval Váš login a (případně zkrácené) téma práce. Vyhněte se však použití mezer, diakritiky a speciálních znaků. Vhodný název může být např.: \uv{\texttt{xlogin00-Cisteni-a-extrakce-textu.tex}}. K přejmenování můžete využít i přiložený Makefile:
\begin{verbatim}
make rename NAME=xlogin00-Cisteni-a-extrakce-textu
\end{verbatim}
  \item Vyplňte požadované položky v souboru, který byl původně pojmenován \texttt{projekt.tex}, tedy typ, rok (odevzdání), název práce, svoje jméno, ústav (dle zadání), tituly a~jméno vedoucího, abstrakt, klíčová slova a další formální náležitosti.
  \item Rozšířený abstrakt v češtině lze v šabloně povolit odkomentováním příslušných 2 bloků v souboru \tt fitthesis.cls\rm .
  \item Nahraďte obsah souborů s kapitolami práce, literaturou a přílohami obsahem svojí technické zprávy. Jednotlivé přílohy či kapitoly práce může být výhodné uložit do~samostatných souborů -- rozhodnete-li se pro toto řešení, je doporučeno zachovat konvenci pro názvy souborů, přičemž za číslem bude následovat název kapitoly. 
  \item Nepotřebujete-li přílohy, zakomentujte příslušnou část v \texttt{projekt.tex} a příslušný soubor vyprázdněte či smažte. Nesnažte se prosím vymyslet nějakou neúčelnou přílohu jen proto, aby daný soubor bylo čím naplnit. Vhodnou přílohou může být obsah přiloženého paměťového média.
  \item Nascanované zadání uložte do souboru \texttt{zadani.pdf} a povolte jeho vložení do práce parametrem šablony v projekt.tex (\verb|documentclass[zadani]{fitthesis}|).
  \item Nechcete-li odkazy tisknout barevně (tedy červený obsah -- bez konzultace s vedoucím nedoporučuji), budete pro tisk vytvářet druhé PDF s tím, že nastavíte parametr šablony pro tisk: (\verb|documentclass[zadani,print]{fitthesis}|).  Barevné logo se nesmí tisknout černobíle!
  \item Vzor desek, do kterých bude práce vyvázána, si vygenerujte v informačním systému fakulty u zadání. Pro disertační práci lze zapnout parametrem v šabloně (více naleznete v souboru fitthesis.cls).
  \item Nezapomeňte, že zdrojové soubory i (obě verze) PDF musíte odevzdat na CD či jiném médiu přiloženém k technické zprávě.
\end{enumerate}

Obsah práce se generuje standardním příkazem \tt \textbackslash tableofcontents \rm (zahrnut v šabloně). Přílohy jsou v něm uvedeny úmyslně.

\subsection*{Pokyny pro oboustranný tisk}
\begin{itemize}
\item \textbf{Oboustranný tisk je doporučeno konzultovat s vedoucím práce.}
\item Je-li práce tištěna oboustranně a její tloušťka je menší než tloušťka desek, nevypadá to dobře.
\item Zapíná se parametrem šablony: \verb|\documentclass[twoside]{fitthesis}|
\item Po vytištění oboustranného listu zkontrolujte, zda je při prosvícení sazební obrazec na obou stranách na stejné pozici. Méně kvalitní tiskárny s duplexní jednotkou mají často posun o 1--3 mm. Toto může být u některých tiskáren řešitelné tak, že vytisknete nejprve liché stránky, pak je dáte do stejného zásobníku a vytisknete sudé.
\item Za titulním listem, obsahem, literaturou, úvodním listem příloh, seznamem příloh a případnými dalšími seznamy je třeba nechat volnou stránku, aby následující část začínala na liché stránce (\textbackslash cleardoublepage).
\item  Konečný výsledek je nutné pečlivě překontrolovat.
\end{itemize}

\subsection*{Styl odstavců}

Odstavce se zarovnávají do bloku a pro jejich formátování existuje více metod. U papírové literatury je častá metoda s~použitím odstavcové zarážky, kdy se u~jednotlivých odstavců textu odsazuje první řádek odstavce asi o~jeden až dva čtverčíky (vždy o~stejnou, předem zvolenou hodnotu), tedy přibližně o~dvě šířky velkého písmene M základního textu. Poslední řádek předchozího odstavce a~první řádek následujícího odstavce se v~takovém případě neoddělují svislou mezerou. Proklad mezi těmito řádky je stejný jako proklad mezi řádky uvnitř odstavce. \cite{fitWeb} Další metodou je odsazení odstavců, které je časté u elektronické sazby textů. První řádek odstavce se při této metodě neodsazuje a mezi odstavce se vkládá vertikální mezera o~velikosti 1/2 řádku. Obě metody lze v kvalifikační práci použít, nicméně často je vhodnější druhá z uvedených metod. Metody není vhodné kombinovat.

Jeden z výše uvedených způsobů je v šabloně nastaven jako výchozí, druhý můžete zvolit parametrem šablony \uv{\tt odsaz\rm }.

\subsection*{Užitečné nástroje}
\label{nastroje}

Následující seznam není výčtem všech využitelných nástrojů. Máte-li vyzkoušený osvědčený nástroj, neváhejte jej využít. Pokud však nevíte, který nástroj si zvolit, můžete zvážit některý z následujících:

\begin{description}
	\item[\href{http://miktex.org/download}{MikTeX}] \LaTeX{} pro Windows -- distribuce s jednoduchou instalací a vynikající automatizací stahování balíčků.
	\item[\href{http://texstudio.sourceforge.net/}{TeXstudio}] Přenositelné opensource GUI pro \LaTeX{}.  Ctrl+klik umožňuje přepínat mezi zdrojovým textem a PDF. Má integrovanou kontrolu pravopisu, zvýraznění syntaxe apod. Pro jeho využití je nejprve potřeba nainstalovat MikTeX.
	\item[\href{http://www.winedt.com/}{WinEdt}] Ve Windows je dobrá kombinace WinEdt + MiKTeX. WinEdt je GUI pro Windows, pro jehož využití je nejprve potřeba nainstalovat \href{http://miktex.org/download}{MikTeX} či \href{http://www.tug.org/texlive/}{TeX Live}. 
	\item[\href{http://kile.sourceforge.net/}{Kile}] Editor pro desktopové prostředí KDE (Linux). Umožňuje živé zobrazení náhledu. Pro jeho využití je potřeba mít nainstalovaný \href{http://www.tug.org/texlive/}{TeX Live} a Okular. 
	\item[\href{http://jabref.sourceforge.net/download.php}{JabRef}] Pěkný a jednoduchý program v Javě pro správu souborů s bibliografií (literaturou). Není potřeba se nic učit -- poskytuje jednoduché okno a formulář pro editaci položek.
	\item[\href{https://inkscape.org/en/download/}{InkScape}] Přenositelný opensource editor vektorové grafiky (SVG i PDF). Vynikající nástroj pro tvorbu obrázků do odborného textu. Jeho ovládnutí je obtížnější, ale výsledky stojí za to.
	\item[\href{https://git-scm.com/}{GIT}] Vynikající pro týmovou spolupráci na projektech, ale může výrazně pomoci i jednomu autorovi. Umožňuje jednoduché verzování, zálohování a přenášení mezi více počítači.
	\item[\href{http://www.overleaf.com/}{Overleaf}] Online nástroj pro \LaTeX{}. Přímo zobrazuje náhled a umožňuje jednoduchou spolupráci (vedoucí může průběžně sledovat psaní práce), vyhledávání ve zdrojovém textu kliknutím do PDF, kontrolu pravopisu apod. Zdarma jej však lze využít pouze s určitými omezeními (někomu stačí na disertaci, jiný na ně může narazit i při psaní bakalářské práce) a pro dlouhé texty je pomalejší.
\end{description}

Pozn.: Overleaf nepoužívá Makefile v šabloně -- aby překlad fungoval, je nutné kliknout pravým tlačítkem na \tt projekt.tex \rm a zvolit \uv{Set as Main File}.


\subsection*{Užitečné balíčky pro \LaTeX}

Studenti při sazbě textu často řeší stejné problémy. Některé z nich lze vyřešit následujícími balíčky pro \LaTeX:

\begin{itemize}
  \item \verb|amsmath| -- rozšířené možnosti sazby rovnic,
  \item \verb|float, afterpage, placeins| -- úprava umístění obrázků,
  \item \verb|fancyvrb, alltt| -- úpravy vlastností prostředí Verbatim, 
  \item \verb|makecell| -- rozšíření možností tabulek,
  \item \verb|pdflscape, rotating| -- natočení stránky o 90 stupňů (pro obrázek či tabulku),
  \item \verb|hyphenat| -- úpravy dělení slov,
  \item \verb|picture, epic, eepic| -- přímé kreslení obrázků.
\end{itemize}

Některé balíčky jsou využity přímo v šabloně (v dolní části souboru fitthesis.cls). Nahlédnutí do jejich dokumentace může být rovněž užitečné.

Sloupec tabulky zarovnaný vlevo s pevnou šířkou je v šabloně definovaný \uv{L} (používá se jako \uv{p}).

 % viz. prilohy.tex / see prilohy.tex
\end{document}
