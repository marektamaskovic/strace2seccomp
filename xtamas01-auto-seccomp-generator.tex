%==============================================================================
% tento soubor pouzijte jako zaklad
% this file should be used as a base for the thesis
% Autoři / Authors: 2008 Michal Bidlo, 2016 Jaroslav Dytrych
% Kontakt pro dotazy a připomínky: dytrych@fit.vutbr.cz
% Contact for questions and comments: dytrych@fit.vutbr.cz
%==============================================================================
% kodovaní: UTF-8 (zmena prikazem iconv, recode nebo cstocs)
% encoding: UTF-8 (you can change it by command iconv, recode or cstocs)
%------------------------------------------------------------------------------
% zpracování / processing: make, make pdf, make clean
%==============================================================================
% Soubory, které je nutné upravit: / Files which have to be edited:
%   xtamas01-auto-seccomp-generator-20-literatura-bibliography.bib - literatura / bibliography
%   xtamas01-auto-seccomp-generator-01-kapitoly-chapters.tex - obsah práce / the thesis content
%   xtamas01-auto-seccomp-generator-30-prilohy-appendices.tex - přílohy / appendices
%==============================================================================
\documentclass[english]{fitthesis} % bez zadání - pro začátek práce, aby nebyl problém s překladem
%\documentclass[english]{fitthesis} % without assignment - for the work start to avoid compilation problem
%\documentclass[zadani]{fitthesis} % odevzdani do wisu - odkazy jsou barevné
%\documentclass[english,zadani]{fitthesis} % for submission to the IS FIT - links are color
%\documentclass[zadani,print]{fitthesis} % pro tisk - odkazy jsou černé
%\documentclass[zadani,cprint]{fitthesis} % pro barevný tisk - odkazy jsou černé, znak VUT barevný
%\documentclass[english,zadani,print]{fitthesis} % for the color print - links are black
%\documentclass[english,zadani,cprint]{fitthesis} % for the print - links are black, logo is color
% * Je-li prace psana v anglickem jazyce, je zapotrebi u tridy pouzit
%   parametr english nasledovne:
%   If thesis is written in english, it is necessary to use
%   parameter english as follows:
%      \documentclass[english]{fitthesis}
% * Je-li prace psana ve slovenskem jazyce, je zapotrebi u tridy pouzit
%   parametr slovak nasledovne:
%      \documentclass[slovak]{fitthesis}

% Základní balíčky jsou dole v souboru šablony fitthesis.cls
% Basic packages are at the bottom of template file fitthesis.cls
%zde muzeme vlozit vlastni balicky / you can place own packages here

%---rm---------------
\renewcommand{\rmdefault}{lmr}%zavede Latin Modern Roman jako rm / set Latin Modern Roman as rm
%---sf---------------
\renewcommand{\sfdefault}{qhv}%zavede TeX Gyre Heros jako sf
%---tt------------
\renewcommand{\ttdefault}{lmtt}% zavede Latin Modern tt jako tt

% vypne funkci šablony, která automaticky nahrazuje uvozovky,
% aby nebyly prováděny nevhodné náhrady v popisech API apod.
% disables function of the template which replaces quotation marks
% to avoid unnecessary replacements in the API descriptions etc.
\csdoublequotesoff

% =======================================================================
% balíček "hyperref" vytváří klikací odkazy v pdf, pokud tedy použijeme pdflatex
% problém je, že balíček hyperref musí být uveden jako poslední, takže nemůže
% být v šabloně
% "hyperref" package create clickable links in pdf if you are using pdflatex.
% Problem is that this package have to be introduced as the last one so it
% can not be placed in the template file.
\ifWis
\ifx\pdfoutput\undefined % nejedeme pod pdflatexem / we are not using pdflatex
\else
  \usepackage{color}
  \usepackage[unicode,colorlinks,hyperindex,plainpages=false,pdftex]{hyperref}
  \definecolor{links}{rgb}{0.4,0.5,0}
  \definecolor{anchors}{rgb}{1,0,0}
  \def\AnchorColor{anchors}
  \def\LinkColor{links}
  \def\pdfBorderAttrs{/Border [0 0 0] }  % bez okrajů kolem odkazů / without margins around links
  \pdfcompresslevel=9
\fi
\else % pro tisk budou odkazy, na které se dá klikat, černé / for the print clickable links will be black
\ifx\pdfoutput\undefined % nejedeme pod pdflatexem / we are not using pdflatex
\else
  \usepackage{color}
  \usepackage[unicode,colorlinks,hyperindex,plainpages=false,pdftex,urlcolor=black,linkcolor=black,citecolor=black]{hyperref}
  \definecolor{links}{rgb}{0,0,0}
  \definecolor{anchors}{rgb}{0,0,0}
  \def\AnchorColor{anchors}
  \def\LinkColor{links}
  \def\pdfBorderAttrs{/Border [0 0 0] } % bez okrajů kolem odkazů / without margins around links
  \pdfcompresslevel=9
\fi
\fi
% Řešení problému, kdy klikací odkazy na obrázky vedou za obrázek
% This solves the problems with links which leads after the picture
\usepackage[all]{hypcap}

% Informace o práci/projektu / Information about the thesis
%---------------------------------------------------------------------------
\projectinfo{
  %Prace / Thesis
  project=BP,            %typ prace BP/SP/DP/DR  / thesis type (SP = term project)
  year=2017,             %rok odevzdání / year of submission
  date=\today,           %datum odevzdani / submission date
  %Nazev prace / thesis title
  title.cs={Automatický generátor politiky systémového volání},  %nazev prace v cestine ci slovenstine (dle zadani) / thesis title in czech language (according to assignment)
  title.en={Automatic Seccomp Syscall Policy Generator}, %nazev prace v anglictine / thesis title in english
  %Autor / Author
  author={Marek Tamaškovič},   %cele jmeno a prijmeni autora / full name and surname of the author
  author.name={Marek},   %jmeno autora (pro citaci) / author name (for reference)
  author.surname={Tamaškovič},   %prijmeni autora (pro citaci) / author surname (for reference)
  %author.title.p=Bc., %titul pred jmenem (nepovinne) / title before the name (optional)
  %author.title.a=PhD, %titul za jmenem (nepovinne) / title after the name (optional)
  %Ustav / Department
  department=UITS, % doplnte prislusnou zkratku dle ustavu na zadani: UPSY/UIFS/UITS/UPGM
  %                  fill in appropriate abbreviation of the department according to assignment: UPSY/UIFS/UITS/UPGM
  %Skolitel / supervisor
  supervisor=Lenka Turoňová, %cele jmeno a prijmeni skolitele / full name and surname of the supervisor
  supervisor.name={Lenka},   %jmeno skolitele (pro citaci) / supervisor name (for reference)
  supervisor.surname={Turoňová},   %prijmeni skolitele (pro citaci) / supervisor surname (for reference)
  supervisor.title.p=Ing.,   %titul pred jmenem (nepovinne) / title before the name (optional)
  supervisor.title.a={},    %titul za jmenem (nepovinne) / title after the name (optional)
  %Klicova slova, abstrakty, prohlaseni a podekovani je mozne definovat
  %bud pomoci nasledujicich parametru nebo pomoci vyhrazenych maker (viz dale)
  %Keywords, abstracts, declaration and acknowledgement can be defined by following
  %parameters or using dedicated macros (see below)
  %===========================================================================
  %Klicova slova / keywords
  %keywords.cs={Klíčová slova v českém jazyce.}, %klicova slova v ceskem ci slovenskem jazyce
  %                                              keywords in czech or slovak language
  %keywords.en={Klíčová slova v anglickém jazyce.}, %klicova slova v anglickem jazyce / keywords in english
  %Abstract
  %abstract.cs={Výtah (abstrakt) práce v českém jazyce.}, % abstrakt v ceskem ci slovenskem jazyce
  %                                                         abstract in czech or slovak language
  %abstract.en={Výtah (abstrakt) práce v anglickém jazyce.}, % abstrakt v anglickem jazyce / abstract in english
  %Prohlaseni / Declaration
  %declaration={Prohlašuji, že jsem tuto bakalářskou práci vypracoval samostatně pod vedením pana ...},
  %Podekovani (nepovinne) / Acknowledgement (optional)
  %acknowledgment={Zde je možné uvést poděkování vedoucímu práce a těm, kteří poskytli odbornou pomoc.} % nepovinne
  %acknowledgment={Here it is possible to express thanks to the supervisor and to the people which provided professional help.} % optional
}

%Abstrakt (cesky, slovensky ci anglicky) / Abstract (in czech, slovak or english)
\abstract[cs]{Do tohoto odstavce bude zapsán výtah (abstrakt) práce v českém (slovenském) jazyce.}
\abstract[en]{Do tohoto odstavce bude zapsán výtah (abstrakt) práce v anglickém jazyce.}

%Klicova slova (cesky, slovensky ci anglicky) / Keywords (in czech, slovak or english)
\keywords[cs]{Sem budou zapsána jednotlivá klíčová slova v českém (slovenském) jazyce, oddělená čárkami.}
\keywords[en]{Sem budou zapsána jednotlivá klíčová slova v anglickém jazyce, oddělená čárkami.}

%Prohlaseni (u anglicky psane prace anglicky, u slovensky psane prace slovensky)
%Declaration (for thesis in english should be in english)
\declaration{Prohlašuji, že jsem tuto bakalářskou práci vypracoval samostatně pod vedením pana X...
Další informace mi poskytli...
Uvedl jsem všechny literární prameny a publikace, ze kterých jsem čerpal.}

% \declaration{Hereby I declare that this bachelor's thesis was prepared as an original author’s work under the supervision of Mr. X
% The supplementary information was provided by Mr. Y
% All the relevant information sources, which were used during preparation of this thesis, are properly cited and included in the list of references.}

%Podekovani (nepovinne, nejlepe v jazyce prace) / Acknowledgement (optional, ideally in the language of the thesis)
\acknowledgment{V této sekci je možno uvést poděkování vedoucímu práce a těm, kteří poskytli odbornou pomoc
(externí zadavatel, konzultant, apod.).}
%\acknowledgment{Here it is possible to express thanks to the supervisor and to the people which provided professional help
%(external submitter, consultant, etc.).}

% řeší první/poslední řádek odstavce na předchozí/následující stránce
% solves first/last row of the paragraph on the previous/next page
\clubpenalty=10000
\widowpenalty=10000

\usepackage{cleveref}
\usepackage{tikz}
\usepackage{tikz-qtree}
\usepackage{float}
\usepackage{xcolor}
\usepackage{listings}
% \usepackage[obeyFinal]{todonotes}

\usetikzlibrary{arrows,backgrounds,shadows, positioning}

% define layers
    \pgfdeclarelayer{edgelayer}
    \pgfdeclarelayer{nodelayer}
% tell TikZ how to stack them (back to front)
    \pgfsetlayers{nodelayer,main,edgelayer}

\begin{document}
  % Vysazeni titulnich stran / Typesetting of the title pages
  % ----------------------------------------------
  \maketitle
  % Obsah
  % ----------------------------------------------
  \setlength{\parskip}{0pt}

  {\hypersetup{hidelinks}\tableofcontents}

  % Seznam obrazku a tabulek (pokud prace obsahuje velke mnozstvi obrazku, tak se to hodi)
  % List of figures and list of tables (if the thesis contains a lot of pictures, it is good)
  \ifczech
    \renewcommand\listfigurename{Seznam obrázků}
  \fi
  \ifslovak
    \renewcommand\listfigurename{Zoznam obrázkov}
  \fi
  % \listoffigures

  \ifczech
    \renewcommand\listtablename{Seznam tabulek}
  \fi
  \ifslovak
    \renewcommand\listtablename{Zoznam tabuliek}
  \fi
  % \listoftables

  \ifODSAZ
    \setlength{\parskip}{0.5\bigskipamount}
  \else
    \setlength{\parskip}{0pt}
  \fi

  % vynechani stranky v oboustrannem rezimu
  % Skip the page in the two-sided mode
  \iftwoside
    \cleardoublepage
  \fi

  % Text prace / Thesis text
  % ----------------------------------------------
  %=========================================================================
% (c) Michal Bidlo, Bohuslav Křena, 2008

\chapter{Introduction}
Nowadays, when malicious code or malware is becoming more and more sophisticated and pressing security risk, it is really needed to control a program behaviour and monitor what the program is doing in a system.
Monitoring program behaviour can be done in many ways and one of the easiest ways is to use Intrusion Detection System (IDS).
IDS is an out-of-the-box solution which can monitor i.e. where program wrote or read something and it is not allowed.
After that, IDS is reporting this violation. %TODO and and

Another way is to monitor and block system calls (syscalls).
Monitoring is performed using tools mentioned in the next chapter.
The actual blocking can be performed with mandatory control access (MAC) (Apparmor, SELinux), sandboxing (seccomp) or others mechanisms.
MAC refers to a type of access control by which the operating system constraints the ability of a subject or initiator to access or generally perform some sort of operation on an object or target.
Seccomp is a Linux kernel module which allows to a process one-way transition to secure a state where the process can only use four syscalls.
When the process tries to call another syscall then one of the four members set is terminated with SIGKILL.
The set of allowed system calls can be extended using seccomp-bpf.
This extension allows filtering system calls using a configurable policy implemented with Berkley Packet Filter (BPF) rules.
This last part is an area on which I would like to focus in my thesis.

I aim to design and develop a tool which helps developers using libseccomp and seccomp-bpf.
I plan to create policies for a specific program in a format readable by libseccomp or seccomp-bpf.

Chapter 1 \todo{add reference to the chapter} describes syscalls and how to monitor them.
In the next chapter of the thesis, I will illustrate how security facilities in Linux, such as systrace and seccomp, work.
After the theoretical part, the design and development of a tool will follow.
In conclusion, methodology how this tool was tested is described.


\chapter{System calls and tools monitoring them}
In this chapter, I will describe a term system call and I will make an overview of tools which can monitor the system calls.
We will red focus in detail on a strace tool which will be used as an input to my tool.
The other tools are described briefly not as detailed as strace.

\section{System calls}
System calls or syscalls is a mechanism used by processes to use operating system functions typically in monolithic kernels.
We can find them on every UNIX system.
In computer terminology, the term syscall is a programmatic way in which a computer program requests a service from the kernel of the operating system it is executed on.

%The system calls are generated using interrupt i.e. on Linux/i86 with interrupt no. 0x80. These interrupts are handled by ...

\section{Monitoring}
\subsection{Strace -- trace system calls and signals}
\subsection{Ftrace -- trace system calls, function calls and signals}
\subsection{Ltrace}
\subsection{Dtrace}
\subsection{Autrace -- linux audit}

\chapter{Security facility in Linux}
Systrace and seccomp\cite{Pravidla}
\section{Systrace}
123

\section{Seccomp}
321
\subsection{Seccomp-bpf}

\subsection{Berkeley packet filter}
\subsubsection{Classic BPF}
\subsubsection{Extended BPF}

\subsection{libseccomp}


\chapter{Development of TODO: tool-name}
\section{Input}
\section{Intermediate representation}
\section{Output}
\section{Heuristics and optimizations}
\subsection{Minimax}
\subsection{Strict}
\subsection{Smart}


\chapter{Testing methods}


%=========================================================================
 % viz. obsah.tex / see obsah.tex

  % Pouzita literatura / Bibliography
  % ----------------------------------------------
\ifslovak
  \makeatletter
  \def\@openbib@code{\addcontentsline{toc}{chapter}{Literatúra}}
  \makeatother
  \bibliographystyle{bib-styles/czechiso}
\else
  \ifczech
    \makeatletter
    \def\@openbib@code{\addcontentsline{toc}{chapter}{Literatura}}
    \makeatother
    \bibliographystyle{bib-styles/czechiso}
  \else
    \makeatletter
    \def\@openbib@code{\addcontentsline{toc}{chapter}{Bibliography}}
    \makeatother
    \bibliographystyle{bib-styles/englishiso}
  %  \bibliographystyle{alpha}
  \fi
\fi
  \begin{flushleft}
  \bibliography{xtamas01-auto-seccomp-generator-20-literatura-bibliography}
  \end{flushleft}

  % vynechani stranky v oboustrannem rezimu
  % Skip the page in the two-sided mode
  \iftwoside
    \cleardoublepage
  \fi

  % Prilohy / Appendices
  % ---------------------------------------------
  \appendix
\ifczech
  \renewcommand{\appendixpagename}{Přílohy}
  \renewcommand{\appendixtocname}{Přílohy}
  \renewcommand{\appendixname}{Příloha}
\fi
\ifslovak
  \renewcommand{\appendixpagename}{Prílohy}
  \renewcommand{\appendixtocname}{Prílohy}
  \renewcommand{\appendixname}{Príloha}
\fi
%  \appendixpage

% vynechani stranky v oboustrannem rezimu
% Skip the page in the two-sided mode
%\iftwoside
%  \cleardoublepage
%\fi

\ifslovak
%  \section*{Zoznam príloh}
%  \addcontentsline{toc}{section}{Zoznam príloh}
\else
  \ifczech
%    \section*{Seznam příloh}
%    \addcontentsline{toc}{section}{Seznam příloh}
  \else
%    \section*{List of Appendices}
%    \addcontentsline{toc}{section}{List of Appendices}
  \fi
\fi
  \startcontents[chapters]
  \setlength{\parskip}{0pt}
  % seznam příloh / list of appendices
  % \printcontents[chapters]{l}{0}{\setcounter{tocdepth}{2}}

  \ifODSAZ
    \setlength{\parskip}{0.5\bigskipamount}
  \else
    \setlength{\parskip}{0pt}
  \fi

  % vynechani stranky v oboustrannem rezimu
  \iftwoside
    \cleardoublepage
  \fi
  \chapter{Comparison of libseccomp and raw BPF filtering}
\lstdefinestyle{c_style}{
    basicstyle=\footnotesize,
    breakatwhitespace=false,         
    breaklines=true,                 
    captionpos=b,                    
    keepspaces=true,                 
    numbers=left,                    
    numbersep=5pt,                  
    showspaces=false,                
    showstringspaces=false,
    showtabs=false,                  
    tabsize=2,
    morecomment=[l]{//},
    morecomment=[s]{/*}{*/},
    morestring=[b]",
    commentstyle=\color{codegreen},
}
\lstset{style=c_style}
\section{BPF}
\lstinputlisting[language=C, label={lst:raw_bpf}, caption={Using raw BPF filtering}]{source_examples/raw_bpf.c}
\section{libseccomp}
\lstinputlisting[language=C, label={lst:libseccomp}, caption={Using simpler libseccomp wrapper}]{source_examples/libseccomp.c}

\chapter{other appendix} % viz. prilohy.tex / see prilohy.tex
\end{document}
